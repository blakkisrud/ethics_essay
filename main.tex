\documentclass[12p]{article}       
\usepackage[utf8]{inputenc}
\renewcommand{\baselinestretch}{1.5}
\usepackage{parskip}
\usepackage{graphicx}       				% For å inkludere figurer.
\usepackage{amsmath,amssymb} 				% Ekstra matematikkfunksjoner.
\usepackage{siunitx}% Må inkluderes for blant annet å få tilgang til kommandoen \SI (korrekte måltall med enheter)
\usepackage{float}
\usepackage{geometry}
%\usepackage{natbib}
%\usepackage{tabular}


\def\ps@pprintTitle{%
\let\@oddhead\@empty
\let\@evenhead\@empty
\def\@oddfoot{}%
\let\@evenfoot\@oddfoot}

%	\sisetup{exponent-product = \cdot}      	% Prikk som multiplikasjonstegn (i steden for kryss).
% 	\sisetup{output-decimal-marker  =  {,}} 	% Komma som desimalskilletegn (i steden for punktum).
 %	\sisetup{separate-uncertainty = true}   	% Pluss-minus-form på usikkerhet (i steden for parentes). 
\usepackage{booktabs}                     		% For å få tilgang til finere linjer (til bruk i tabeller og slikt).
\usepackage[font=small,labelfont=bf]{caption}		% For justering av figurtekst og tabelltekst.

% Disse kommandoene kan gjøre det enklere for LaTeX å plassere figurer og tabeller der du ønsker.
\setcounter{totalnumber}{5}
\renewcommand{\textfraction}{0.05}
\renewcommand{\topfraction}{0.95}
\renewcommand{\bottomfraction}{0.95}
\renewcommand{\floatpagefraction}{0.35}
\geometry{margin = 0.75in}

\title{Title 1: Randomized controlled trials \\ \large{An introduction, a story, two pre-requisites, a dilemma, an attempt on justification, a (small personal) problem, a reality check and a middle ground}
  %\footnote{Strongly depends of what I reach, "beauty flaws", "a false cry of saving", "without the means to end all discussion"}
}
\author{Johan Blakkisrud}

\begin{document}

\maketitle

\section*{Aim}

"Randomized controlled trials appear to annoy human nature - if properly conducted, indeed they should." - Schultz 1995

Randomized Controlled Trials (RCTs) is by a majority of the medical community considered as the "gold standard" for medical research.
This essay started as out a comprehensive investigation of the however this god-standard study-design also had an ethically sound foundation.
I painted myself into the proverbial corner trying to answer the question, and was (by my own free will) obligated to shorten it down.
I then decided to present eight parts, each musing on the nature of RCTs.
The parts are an introduction, a story, two definitions, a dilemma, an attempt on justification, a (small personal) problem, two extremes, a reality check and a middle ground.

\section*{Introduction}

A randomized control trial (RCT), sometimes also refered to as a randomized clinical trial or a randomized controlled trial, is a commonly used study design in clinical trials.
A fundamental idea of the design is to group the patients prior to the study into (most commonly) two the random selection of patients into these groups are the "randomized controlled" part.
The patients are then treated according to their group.
Typically one group receives a novel treatment, and the outcome of the group is compared to a "status quo treatment".
Another typical example is comparing the treatment to a group given a placebo. 
If the two groups of patients are equal, the effect of the better drug should reveal itself.

Today, in and around the medical community, such a design is the gold standard.
It is the study design said to be the design of which all other designs should try to replicate. 
A search for "Clinical" and "Randomized Controlled Trial" as a publication type in the database of US National Library of Medicine (PUBMED) gave on the 20. of June 2017 308 669 results, numbers gradually increasing since XX
Another role of RCTs is that they are gate-keepers of new drugs, which has to pas one (or sometimes several) RCTs to become avaliable on the market. 
For being such a large and leading part of the medical field, the ethical considerations should also stand to great scrutiny.

\section*{A story}

The idea is not new, the history of RCTs begins in the 18th century, during the age of sail.
Long voyages and monotenous food-supply on navy vessels resulted in a heinos condition, costing an estimated 2 million lives between 1500 to 1800, scurvy.
Documentation of the symptoms of scurvy dates back to Hippocrates, and symptoms of a scurvy-like disease was recorded by the ancient egyptians some 3500 years ago.
A cure however, had eluded man for centuries \footnote{Some honorable mentions goes to Jaques Cartier who learned to drink water boiled with Eastern White Cedar, Sir Richard Hawkins who recomended orange and lemon juice, and John Woodall that recommended fresh fruits in general}
This changed in 1747, when James Lind, a Scottish physician, proved that sailors drinking citrus was spared for the disease.
%The interesting thing with Linden is that he  [did split patients into groups, and gave different remedies]
The detail that separates Lind from the others, is that instead of setting out to prove that a specific remedy helped, he carefully selected 12 patients, gave two and two different treatments, and watched the outcome.
Patients were selected for the group to be is homogeneous as possible in respect to severity of disease, diet etc.
The common diet in itself is interesting "...water gruel sweetened with sugar...fresh mutton broth...boiled biscuit with sugars" and "barley and raisins, rice and currents, sago and wine and the like" \cite{RN3}.
Two patients took "elixir vitriol" thrice a day, two others got vinegar, two got cider, two got an "electory recommended by the surgeon general", two got seawater (!) and the last two got two lemons and one lime each day.
All patients apart from the two citrus-consumers (and to a lesser extent the cider-drinkers) deteoriated and Lind concluded that citrus was the best remedy.
These weeks in the ultimo of May 1747 changed medicine.

%A full covering of the ethical concerns of RCTs is beyond the scope of a meager essay.
%I will focus my discussion on XX concerns, first "The dilemma of the physician-scientist", secondly I will highlight a problem with 

%In this essay I will try to highlight some aspects of RCTs in general, describing and dealing with concepts such as equipose, blinding, written concent and so on.
%I will then discuss these elements of RCTs as well as RCTs in general within the most common ethical frame works, first and foremost utilitarism and deontology. 
%Lastly I will try to sum up with some concluding remarks.

\section*{Two requirements (meaningful?)}

Literature around the ethical concerns of RCTs often notes a concept called equipoise, also called "clinical equipoise".
The term was coined by Benjamin Freedman in 1987 [cite].
It is used as a description of a situation where there is an uncertainty if way of treatment is more effective than the other.
Equipoise is said to negate the dilemma where a patient may end up in a situation where the patient is worse of than with the "standard" treatment.
Clinical equipoise then becomes a \emph{necessary} part of a ethical clinical trial.

The second requirement is written consent, more precisely informed written consent.
The process of obtaining written consent is in a large number of countries formalized and tightly regulated through review boards and ethical committees.
The consent must also be "informed", meaning that the patients must have an understanding of what the risk and benefits are.
The need for informed consent stems from a time where the patients autonomy was a smaller concern.
Although written consent was part of medical research early on, they were more to "ensuring that the subject's compliance rather than an expression of concern for their welfare" [cite Nadini].

%It has been argued that such consent forms should \emph{not} be too detailed, as it can strongly influence patients decisions. 

\section*{A dilemma}

Now is a good time to introduce some kind of practical example, grounding our discussion in some borderline real-life situations.
A physician is conducting a trial, testing a new form of monoclonal antibodies to treat an especially nasty type of cancer.
The new treatment is though to have mild side effects, as shown in a small prior cohort of cancer patients, compared to the standard treatment (several rounds of chemotherapy).
Now a RCT is planned and executed, with 50 patients receiving the new treatment and 50 patients receiving the standard treatment.
All patients signed a written form of consent and were randomized into each group according to established good practises.
As we will make ourself omniscient for the sake of argument, lets assume that the new treatment is \emph{vastly} better than the standard treatment, both in curing and increase the quality of life of the patients.

%As the number, range and general use of RCTs are so enormous, I will limit the discussion as follows:
%First of all, I will limit my discussion to medical research.
%Interesting concerns are raised in regards to the use of animals, but I will focus on cases where humans are the subjects.
%I will also focus on research and treatments that are "invasive", meeting one or more of these four conditions:
%(a) introduces a foreign element through surgery 
%(b) introduces a pharmaceutical with an expected effect
%(c) uses ionizing radiation
%(d) any procedure that can cause distress in a normal human being

Now, the dilemma in the title belongs to the person conducting the trial.
She has a dilemma.
Lets assume that she is a physician, she would then be a "physician-scientist", a term coined by Hellman and Hellman [cite].
The term is rather apt, although a bit unmusical, describing two roles that are difficult to reconcile.
The physician connects deeply with the patient.
As Hellmann and Hellmann cites Leon Kass: "the physician must produce unswervingly the virtues of loyalty and fidelity to his patient."
%The physician has a duty towards the patient to do good and not cause harm.
It is simple to argue that this relationship should (and to a large extent also is) mandated primarily by deontological ethics.
The physician feel a \emph{moral} obligation to help her patients.
Close (some might argue closer) is it natural to describe the situation in terms of \emph{virtue ethics}.
Beuchamp and Childress, cited in the paper "A virtue ethics approach to moral dilemmas in medicine", named five virtues applicable to the medical practitioner:
Trustworthiness, integrity, discernment, compassion and conscientiousness.
The physician that adheres to these five virtues should fulfill the common notion of a "good doctor".
Either if the main rationale stems from virtue or moral obligation, the physician should do what is best for her patient.
This extends naturally to all patients, and even though the physicians time and energy are limited resources, the "good doctor" would either disperse her time across all patients or refer the patients to another (good) doctor.

Now, the other of the dual roles of our physician comes into play, the scientist of the "physician-scientist".
I have used the word "patient" but I could just as well have used the term "research subject" to further separate the "care from the experiment".
It would be naive and a bit simple to say that the scientist disregard the well being of the patients.
On the contrary, the felt responsibility towards \emph{future} patients is potentially quite strong, with compassion and the intent to do no harm.
But, other virtues and morals comes into play, the scientist also have to be precise, thorough and strive to get results that further our knowledge.
The maxim that the patients well being is the only thing that matters does not apply to the "scientist".
She is still treating, but she is not doing therapy, she is doing research.
%Although it is also natural to describe the scientific effort also in terms of deontological and virtue ethics, we are now moving to a realm where consequence-ethics is arguably more natural.

The outcome of the experiment is unknown for the physician-scientist, it is unknown for the patient and it is unknown for the whole medical community.
Given that the new treatment is fully ineffective, having no treatment ability, we have introduced a lot of suffering, with no benefit.
Conversely, if the treatment is enormously more effective than the current treatment, we introduce suffering by not offering the treatment to the patients in the control arm.
Either way, suffering in the form of sub-optimal treatment is inevitable, it is a fundamental pre-requisite for the trial to have meaning.
The afore-mentioned physician-scientist dilemma now comes full force.
In order to gather medical evidence, the "good doctor" must become less of a good doctor for some of her patients, and fulfilling the role of both physician and scientist becomes difficult.
It is evident that continuing a trial where (perhaps for unforeseeable reasons) a large amount of suffering has been introduced is unethical.
In a utilitarian view this is fairly unproblematic given the payoff in the end, but it goes at the expenses of the integrity of the patients as moral subjects.
For the scientist it could be detrimental to abort the treatment (being the new or old treatment) to soon because then the suffering would be meaningless, we did not learn anything new.


%A solution to this dilemma is not evident.
%One possible angle of attack, is to have a clinical trial lead fifty-fifty by a physician and a scientist.
%This might create more problems than solutions, the internal struggle becomes an external.
%We could remove the physician from the equation entirely, if we argue strong enough that the patients are merely research subjects.
%Patient care during the treatment are done by medical doctors, but the supervision is done by a person that is 

\section{Justification}

Naturally, since RCTs is a large part of the big (and thoroughly ethically scrutinised) field of medical research, justifications are numerous.
The first is very simple:
A smaller group (the research subjects) are sacrificing themselves for some "greater good", the potential of a better cure.
The number of patients in the trial is assumed to be much smaller than the number of people benefiting from the trial.
This is text-book utilitarianism.%, even though the treatment would be deemed in-effective, the medical knowledge has been expanded.
It is then tempting to accept RCTs based on consequencialism alone.
But this is not the only justification.
If true equipoise exist, there is also a moral obligation associated with the patients.
It is a noble act to find the best cure, and an obligation to contribute to the society of which the individual have reaped a lot of benefit from [cite platon].
There is also an interesting historical notion, as the people as of now has benefited from RCTs conducted in the past, we should contribute to the future.

The last justification has a weakness of not justifying the first RCT.
However, as we live in a world where RCTs already occurred and have benefited humankind, the pragmatical thing to do is to accept it.
The moral obligation is also strong, we should do it ourself since we all to larger or lesser degree have a higher quality of life and spared suffering from other individuals that did it.

This is not to say that we should all sacrifice our self on the scientific alter.
We could construct a veil of ignorance, stretching so long that it covers both the past and future.
It is then a bit harder to accept that you should put your self (and others) in harms way.
Given the previous "old treatment" (interestingly enough the individuals has no idea how the treatment was found) and the new.
If we assume that real clinical equipoise exist, a person suffering from the cancer would happy to participate in the study, he will get a treatment.
What is stopping him is the extra hassle with the non-clinically monitoring.
If he is in a state of true clinical equipoise the logical thing to do is to opt for the old treatment, the one that comes without the hassle.
This is of course if he do not believe that partaking in the trial on given a 50-50 chance for what he \emph{might} believe is a better treatment.
Believe is here the key-word, if the patient is believing that a treatment is better than the other, he is strictly no longer in clinical equipoise.
In my opinion, the RCTs do not survive under a veil of ignorance where there is no concept of past and present.
Ideas of past and future generations is necessary for patients to partake in RCTs on good ethical foundations based on deontologic or virtue ethics.
Luckily the RCTs only have meaning in a world we have these concepts.

The first argument is perhaps the strongest.
RCTs maximises utility, and if one accepts utility as the ethical right thing to do, one should have little problem to accept RCTs.
However, there are some problems, and a personal problem I have is described next in:

%the duty towards the expansion of knowledge and virtues governing good scientific conduct (honesty, curiosity perseverance etc. [cite Nature])

\section*{a (small and personal) problem}

Consequencialism relies on the consequences of actions.
But, when does the consequence arise?
Immediately after the action?
A lot of actions are merely chains of events, where one "action" blends into another "action".
When the chain of actions becomes long, the influence of each action becomes smaller, and the endpoints on the chain becomes rather arbitrarily.
Where does the chain end?
To illustrate we might do a more elaborate though experiment:
A researcher in Brazil is doing a large RCT on patients suffering from our aforementioned nasty cancer.
A parallel study is also conducted in Mexico, and is starting to recruit patients when the Brazilian study is well under way.
Now, the patients in the Brazilian study are having bad side-effects of the new drug, and the study is cancelled.
The Mexican physician-scientist is also cancelling his experiment, feeling that it is unethical to risk the well being of the patients.
The cancellations are ethically sound, sparing the patients harm from the seemingly worse treatment.
But, in the land of  Brutopia, a Brutopian physician-scientist does a trial anyway, driven by her own ideas and relaxed regulations.
The trial goes very well in favor for the new therapy, with strong positive response when compared to the previous therapy.
Some years of research down the line, it becomes clear that the reason for the severe side-effects in the Brazilians were due to a rare gene shared by many Brazilians, but almost none Brutopians.
The lessons learned from the Brutopians were shared to the rest of the world community and appropriate patients (lacking the Brazilian gene) were treated and suffering in the world was reduced.
Now, the Mexican physician-scientist did the ethical right thing, both in light of moral ethics and consequencialism.
But in hindsight, so did the Brutopian, although unjustified at the time it is according to consequencialism retrospective ethically sound.
Further more, this argumentation could be used to justify fluke experiments if it could be pointed to other fluke experiments retrospectively shown themselves to yield good (and ethically sound) results.

One could argue that such a scenario is highly unlikely and quite "artificial" and contrived. % but there are numerous examples where odd coincidences and "stray findings" have contributed a lot to the corpus of medical knowledge.
But I would argue that it showcases a problem with "pure" consequencialism, knowing \emph{when} and \emph{where} to stop and calculate the tally of positive and negative consequences is notoriously difficult.
The very least we must accept, is that our chain of actions and consequences can be broken down in small incremental steps, and each step yields a sum of "good" consequences.
It is then possible to ethically accept in a step-by-step (or link-by-link)-manner.
There is still a problem though the chain is in the grand scheme of thing bigger than its parts, and an unacceptable  negative consequence can come from a chain where the individual parts are believed to yield acceptable consequences.
Of course the opposite is also true, as in our overly complicated second though-experiment.
If we have a proper idea of the consequences of the link, it could be argued that that this is all that we can ask for, after all we could not demand our physician-scientists to be omniscient of all foreseeable consequences, then the trial itself would be superfluous.
Our duty then becomes to make the links in the chain smaller and smaller.

%Only then is RCTs (or any kind of clinical trials for that matter) justified.


\section*{A reality check}

This drifted a bit out of hand.
In reality a somewhat solid grasp of the consequences is in place.
The reason for this is that the RCT is often the last link in a bigger chain of experiments. 
Staring in the petri-dish, trough mice and finally in a few human subjects.
Something interesting in RCTs grand scheme of things, is as mentioned in the opening paragraph, that it is the "gold standard".
This is also sometimes used as an argument for RCTs, that without them we must go to the alternative (doctor hunches, anecdotal evidence) [cite Of mice but not men].
But, for the RCTs to be ethically sound, in addition to the previously required (but not sufficient) real clinical equipoise and written consent I would argue that we absolutely need the previous studies to make the links in the chain small.
Further more, with the rise of "personalized medicine" where the treatment to an illness is to a large degree tailored to the patient, the study design of the RCTs becomes unfit, and alternatives are needed.

The physician-scientist dilemma also has some real consequences.
Although it is easy to argue that the virtues and moral of the scientist should trump those of the physician in the clinical trial, that is rather difficult.
Solid evidence and numerous anecdota evidence show that doctors tend to cheat when they are conduction exeperiments, showcased in the marvelous article "Subverting randomization in Controlled trials" [cite]
This cheating consists of subverting the randomization process by putting patients into arms they \emph{know} is better for the patient.
This is not done (or is at least not said to be done) maliciously by the physicians, at least in every case.
It is the physician acting on a belief for what is best for his or her patient.
These notions is difficult to weed off, after all, they stem from basic human nature of not doing harm.

\section*{Broadening of equipoise?}

%Consider a world where we do not known have to transplant kidneys, a relatively simple procedure but merely 60 years old.
%We are interested to see if a kidney can be transplanted from a healthy, living human to a human in need of a kidney.
%We 
%Yes, we now know we can extract a kidney - potentially saving numerous lives.

%A mandatory requirement in almost all clinical trials are a written consent from the patients.
%% This is a rather odd notion, 
%The written consent should remove some of the problems encountered previously.
%The patients \emph{knows} that they risk harm from the treatment and they \emph{know} that they help to increase the knowledge about the disease.
%And they are perfectly \emph{fine} with that, they have, after all, given their written consent. 
%It is the ultimate get-out of jail-free card for the scientist.
%There are obvious problems with it though.
%If we consider the written consent as both necessary and sufficient, we exclude treatment designs on subjects that are not considered to be able to give consent, like children and the severe mentally handicapped.
%
%If the patients wants to sacrifice themselves on the scientific alter, we should let them.
%There are some problems with this as well.
%The validity of written consent relies on the factor the patients have understood the full consequences of their signing.
%This extends understanding the stress and unpleasantness associated with the "extra" non-beneficial interventions (e.g. blood drawing and sitting still for hours in a scanner).
%A more fundamental, and problematic concern is that the patients do not understand that they are \emph{not} treated but experimented on.
%This notion has its own term "therapeutic misconception", coined by Applebaum in the early eighties.
%The patient believes he gets the best treatment.
%This is not however because the patient is stupid.
%Consider for


%This comes on the top of the \emph{known} sufferings accociated with a clinical trial itself, in the form of medically non-beneficial suffering 

%One could argue that the body of medical knowledge has increased (so no one tries that treatment again) but the argument is weak, it do not stop us from conducting fluke experiments with a known control.
%Of course, in the real world checks and balances stops fluke experiments, on the basis 

I have used the terms "better" and "worse" treatments when comparing two treatments being tested.
Implicit I have though of the "better" treatment as the more effective (making the patients free from their illness).
However, we could include a number of parameters making a treatment "better" or "worse".
One is side-effects, if two treatments are equally effective, but one of the treatments result in severe side effects, the other could be thought of as "better".
It is not evident however, what the individual patient consider as a more severe side effects.
If one treatment results in severe dis figuration, and another (equally effective) side effect results in prolonged fatigue, some patients would prefer one that another might not prefer.
Should we, in a state of true equipoise (if there are such a thing) let the patients decide which treatment to take, based on their preference in regards to side effects?
The definition of equipoise concerns as it is stated here the "effectiveness" of the treatment.
Is it then a real equipoise for the patient?
Perhaps in the sense of the "clinical equipoise", but is that sufficient?
Should the term be broadened to include also the patients own preferences of treatment?
%One is likely to say that this introduces bias in the randomization process.
%If some clinically relevant parameter is associated with the aversion for a certain side-effect this can indeed be damaging for the integrity of the randomization.
%However, \emph{some} level of "non-randomization" is a large part of 

%Given a "perfect" patients which understands the full consequences 
%However, it is wrong 
%For the RCTs the situation is a bit complicated.
%The patient has to consent to both treatments, but only receiving one.

%\section*{A middle ground}




%An important aspect in an RCTs is something called "power", or "statistical power".
%One conducts an analysis before the RCT start and calculates the needed participants in the study to reveal an effect.
%The results from an "underpowered" RCT should be considered "scientifically worthless" (Halpern citing Altmann). 
%To call them 
 
%There are numerous advantages that comes with randomization.
%First of all, it eliminates bias by trying to remove systematic differences between groups.
%A second, and a bit more complicated: 
%When we interpret results from clinical research, we use statistical theory which is based on random sampling to infer information of a larger group (in theory all patients) from relatively small group (the research subjects). (cite:altman).

%This is a more "broad" definition of invasive than commonly used.
%The last one is particularly vague and requires individual considerations.
%I will not for the sake of discussion limit the illness or ailment that is treated, for now it could be anything ranging from cosmetic to life-saving.






%\subsection*{Contemporary views}
%How is RCTs generally regarded?
%\subsection*{Contemporary discussions (laymen, the community)}
%What do the media and the general public think?
%\subsection*{Contemporary discussions (medical community)}
%How is regarded by the medical professionals, doctors and other scientists?
%\subsection*{Contemporary discussions (ethical scholars)}
%Are there a philosophical discussion? Have they reached any conclusions? Are they correct?
%\subsection*{In the frameworks of "big" ethical schools}
%Who have the biggest problem with RCTs, and should we care?
%
%
%\section*{Ethical resolutions?}
%\subsection*{Solutions to the problems (if any)}
%Is it all hopeless?
%\section*{Conclusion (possibly vague)}
%
%\section*{Own notes}
%
%\begin{enumerate}
%
%\item
%  The example with the kidney is interesting, as in "is there a difference to take out a kidney 
%  and throw it in the thrash, to investigate of whether or not a kidney "can" be extracted in contrast
%  it to take it out and put it into a needing receipient?
%\item
%  Separation between research and treatment - do we need the treatment aspect as it is not the same - failure to acknowledge that can lead to "very bad things"
%  But, also important, the complete separation can also lead to very bad things
%\item
%  Wacky though: suffering introduced (not death, or, why not death?) for some greater good - can it be drawn a line between "holocaust" and todays practice? 
%  If not, we have a problem.
%  
%  \end{enumerate}
%

\bibliography{test}

  %\begin{thebibliography}{9}
  %
  %  \bibitem{RN2}
  %Bolland, Mark J. and Avenell, Alison and Gamble, Greg D. and Grey, Andrew
  %Systematic review and statistical analysis of the integrity of 33 randomized controlled trials
  %Neurology
  %87
  %23
  %2391-2402
  %2016
  %
  %\bibitem{RN3}
  %Dunn, P.
  %James Lind (1716-94) of Edinburgh and the treatment of scurvy
  %Archives of Disease in Childhood Fetal and Neonatal Edition
  %76
  %1
  %F64-F65
  %1997
  %
  %\bibitem{RN1}
  %Savulescu, Julian and Wartolowska, Karolina and Carr, Andy
  %Randomised placebo-controlled trials of surgery: ethical analysis and guidelines
  %Journal of Medical Ethics
  %42
  %12
  %776-783
  %2016
  %
  %\end{thebibliography}
 

\end{document}

