\documentclass[12p]{article}       
\usepackage[utf8]{inputenc}
\usepackage{parskip}
\usepackage{graphicx}       				% For å inkludere figurer.
\usepackage{amsmath,amssymb} 				% Ekstra matematikkfunksjoner.
\usepackage{siunitx}% Må inkluderes for blant annet å få tilgang til kommandoen \SI (korrekte måltall med enheter)
\usepackage{float}
\usepackage{geometry}
%\usepackage{tabular}


\def\ps@pprintTitle{%
\let\@oddhead\@empty
\let\@evenhead\@empty
\def\@oddfoot{}%
\let\@evenfoot\@oddfoot}

%	\sisetup{exponent-product = \cdot}      	% Prikk som multiplikasjonstegn (i steden for kryss).
% 	\sisetup{output-decimal-marker  =  {,}} 	% Komma som desimalskilletegn (i steden for punktum).
 %	\sisetup{separate-uncertainty = true}   	% Pluss-minus-form på usikkerhet (i steden for parentes). 
\usepackage{booktabs}                     		% For å få tilgang til finere linjer (til bruk i tabeller og slikt).
\usepackage[font=small,labelfont=bf]{caption}		% For justering av figurtekst og tabelltekst.

% Disse kommandoene kan gjøre det enklere for LaTeX å plassere figurer og tabeller der du ønsker.
\setcounter{totalnumber}{5}
\renewcommand{\textfraction}{0.05}
\renewcommand{\topfraction}{0.95}
\renewcommand{\bottomfraction}{0.95}
\renewcommand{\floatpagefraction}{0.35}
\geometry{margin = 0.75in}

\title{Title 1: Randomized controlled trials, the gold standard with?
  \footnote{Strongly depends of what I reach, "beauty flaws", "a false cry of saving", "without the means to end all discussion"}
}
\author{Johan Blakkisrud}

\begin{document}

\maketitle

\section*{Aim}

Randomized Controlled Trials (RCTs) is by a majority of the medical community considered as the "gold standard" for medical research.
This essay is aiming to see if RCTs also are ethically sound.

\section*{Introduction}

\subsection*{Definition}

Definition from Meriam-Webster
\subsection*{Historical origins}
Scurvy and lime juice - aye matey!
\subsection*{Use today}
Numbers, form and distribution, from sugar and salt to surgical intervention.
\subsection*{Limitations}
As the number, range and general use of RCTs are so enourmeus, I will limit the discussion as follows:
First of all, I will limit my discussion to medical research.
Interesting concerns are raised in regards to the use of animals, but I will focus on cases where humans are the subjects.
I will also focus on research and treatments that are "invasive", meating one or more of these four conditions:
(a) introduces a foreign element through surgery 
(b) introduces a pharmaceutical with a known or loosely known effect
(c) uses ionizing of radiation
(d) any procedure that can cause distress in a normal human being
This is a more "broad" definition of invasive than commonly used.
The last one is particularly vague and requires individual considerations.
I will both consider the situation where two treatments are opposed to each other, and situations where one treatment is compared to a placebo.
\section*{Ethical problems?}
\subsection*{Contemporary views}
How is RCTs generally regarded?
\subsection*{Contemporary discussions (laymen, the community)}
What do the media and the general public think?
\subsection*{Contemporary discussions (medical community)}
How is regarded by the medical professionals, doctors and other scientists?
\subsection*{Contemporary discussions (ethical scholars)}
Are there a philosophical discussion? Have they reached any conclusions? Are they correct?
\subsection*{In the frameworks of "big" ethical schools}
Who have the biggest problem with RCTs, and should we care?


\section*{Ethical resolutions?}
\subsection*{Solutions to the problems (if any)}
Is it all hopeless?
\section*{Conclusion (possibly vague)}

\section*{Own notes}

\begin{enumerate}

\item
  The example with the kidney is interesting, as in "is there a difference to take out a kidney 
  and throw it in the thrash, to investigate of whether or not a kidney "can" be extracted in contrast
  it to take it out and put it into a needing receipient?
\item
  Separation between research and treatment - do we need the treatment aspect as it is not the same - failure to acknowledge that can lead to "very bad things"
  But, also important, the complete separation can also lead to very bad things
\item
  Wacky though: suffering introduced (not death, or, why not death?) for some greater good - can it be drawn a line between "holocaust" and todays practice? 
  If not, we have a problem.
  
  \end{enumerate}

\begin{thebibliography}{9}

  \bibitem{RN2}
Bolland, Mark J. and Avenell, Alison and Gamble, Greg D. and Grey, Andrew
Systematic review and statistical analysis of the integrity of 33 randomized controlled trials
Neurology
87
23
2391-2402
2016

\bibitem{RN3}
Dunn, P.
James Lind (1716-94) of Edinburgh and the treatment of scurvy
Archives of Disease in Childhood Fetal and Neonatal Edition
76
1
F64-F65
1997

\bibitem{RN1}
Savulescu, Julian and Wartolowska, Karolina and Carr, Andy
Randomised placebo-controlled trials of surgery: ethical analysis and guidelines
Journal of Medical Ethics
42
12
776-783
2016

\end{thebibliography}
 

\end{document}

