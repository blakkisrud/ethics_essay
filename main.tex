\documentclass[12p]{article}       
\usepackage[utf8]{inputenc}
%\renewcommand{\baselinestretch}{1.5}
\usepackage{parskip}
\usepackage{graphicx}       				% For å inkludere figurer.
\usepackage{amsmath,amssymb} 				% Ekstra matematikkfunksjoner.
\usepackage{siunitx}% Må inkluderes for blant annet å få tilgang til kommandoen \SI (korrekte måltall med enheter)
\usepackage{float}
\usepackage{geometry}
%\usepackage{natbib}
%\usepackage{tabular}


\def\ps@pprintTitle{%
\let\@oddhead\@empty
\let\@evenhead\@empty
\def\@oddfoot{}%
\let\@evenfoot\@oddfoot}

%	\sisetup{exponent-product = \cdot}      	% Prikk som multiplikasjonstegn (i steden for kryss).
% 	\sisetup{output-decimal-marker  =  {,}} 	% Komma som desimalskilletegn (i steden for punktum).
 %	\sisetup{separate-uncertainty = true}   	% Pluss-minus-form på usikkerhet (i steden for parentes). 
\usepackage{booktabs}                     		% For å få tilgang til finere linjer (til bruk i tabeller og slikt).
\usepackage[font=small,labelfont=bf]{caption}		% For justering av figurtekst og tabelltekst.

% Disse kommandoene kan gjøre det enklere for LaTeX å plassere figurer og tabeller der du ønsker.
\setcounter{totalnumber}{5}
\renewcommand{\textfraction}{0.05}
\renewcommand{\topfraction}{0.95}
\renewcommand{\bottomfraction}{0.95}
\renewcommand{\floatpagefraction}{0.35}
\geometry{margin = 0.75in}

\title{Title 1: Randomized controlled trials, the gold standard with?
  %\footnote{Strongly depends of what I reach, "beauty flaws", "a false cry of saving", "without the means to end all discussion"}
}
\author{Johan Blakkisrud}

\begin{document}

\maketitle

\section*{Aim}

"Randomized controlled trials appear to annoy human nature - if properly conducted, indeed they should." - Schultz 1995

Randomized Controlled Trials (RCTs) is by a majority of the medical community considered as the "gold standard" for medical research.
This essay is aiming to see if RCTs also are ethically sound.

\section*{Introduction}

A randomized control trial (RCT), sometimes also refered to as a randomized clinical trial or a randomized controlled trial, is a commonly used study design in clinical trials.
A fundamental idea of the design is to group the patients prior to the study into (most commonly) two the random selection of patients into these groups are the "randomized controlled" part.
The patients are then treated according to their group.
Typically one group receives a novel treatment, and the outcome of the group is compared to a "status quo treatment".
Another typical example is comparing the treatment to a group given a placebo. 
If the two groups of patients are equal, the effect of the better drug should reveal itself.

The idea is not new, the history of RCTs begins in the 18th century, during the age of sail.
Long voyages and monotenous food-supply on navy vessels resulted in a heinos condition, costing an estimated 2 million lives between 1500 to 1800, scurvy.
Documentation of the symptoms of scurvy dates back to Hippocrates, and symptoms of a scurvy-like disease was recorded by the ancient egyptians some 3500 years ago.
A cure however, had eluded man for centuries \footnote{Some honorable mentions goes to Jaques Cartier who learned to drink water boiled with Eastern White Cedar, Sir Richard Hawkins who recomended orange and lemon juice, and John Woodall that recommended fresh fruits in general}
This changed in 1747, when James Lind, a Scottish physician, proved that sailors drinking citrus was spared for the disease.
%The interesting thing with Linden is that he  [did split patients into groups, and gave different remedies]
The detail that separates Lind from the others, is that instead of setting out to prove that a specific remedy helped, he carefully selected 12 patients, gave two and two different treatments, and watched the outcome.
Patients were selected for the group to be is homogeneous as possible in respect to severity of disease, diet etc.
The common diet in itself is interesting "...water gruel sweetened with sugar...fresh mutton broth...boiled biscuit with sugars" and "barley and raisins, rice and currents, sago and wine and the like" \cite{RN3}.
Two patients took "elixir vitriol" thrice a day, two others got vinegar, two got cider, two got an "electory recommended by the surgeon general", two got seawater (!) and the last two got two lemons and one lime each day.
All patients apart from the two citrus-consumers (and to a lesser extent the cider-drinkers) deteoriated and Lind concluded that citrus was the best remedy.
These weeks in the ultimo of May 1747 changed medicine.

Today, in and around the medical community, such a design is the gold standard.
It is the study design said to be the design of which all other designs should try to replicate. 
A search for "Clinical" and "Randomized Controlled Trial" as a publication type in the database of US National Library of Medicine (PUBMED) gave on the 20. of June 2017 308 669 results, numbers gradually increasing since XX
Another role of RCTs is that they are gate-keepers of new drugs, which has to pas one (or sometimes several) RCTs to become avaliable on the market. 
For being such a large and leading part of the medical field, the ethical considerations should also stand to great scrutiny.
In this essay I will try to highlight some aspects of RCTs in general, describing and dealing with concepts such as equipose, blinding, written concent and so on.
I will then discuss these elements of RCTs as well as RCTs in general within the most common ethical frame works, first and foremost utilitarism and deontology. 
Lastly I will try to sum up with some concluding remarks.

Now is a good time to introduce some kind of practical situation, grounding our discussion in some borderline real-life situation.
A physician is conducting a trial, testing a new form of monoclonal antibodies to treat an especially nasty type of cancer.
The new treatment is though to have mild side effects, as shown in a small prior cohort of cancer patients, compared to the standard treatment (several rounds of chemotherapy).
Now a RCT is planned and executed, with 50 patients receiving the new treatment and 50 patients receiving the standard treatment.
All patients signed a written form of consent and were randomized into each group according to established good practises.
As we will make ourself omniscient for the sake of argument, lets assume that the new treatment is \emph{vastly} better than the standard treatment, both in curing and increase the quality of life of the patients.
%As the number, range and general use of RCTs are so enormous, I will limit the discussion as follows:
%First of all, I will limit my discussion to medical research.
%Interesting concerns are raised in regards to the use of animals, but I will focus on cases where humans are the subjects.
%I will also focus on research and treatments that are "invasive", meeting one or more of these four conditions:
%(a) introduces a foreign element through surgery 
%(b) introduces a pharmaceutical with an expected effect
%(c) uses ionizing radiation
%(d) any procedure that can cause distress in a normal human being

The first aspect I want consider, is the person conducting the RCT.
She has a dilemma.
Lets assume that she is a physician, she would then be a "physician-scientist", a termed coined by Hellman and Hellman [cite].
The term is rather apt, although a bit unmusical describing two roles that are difficult to reconcile.
The physician connects deeply with the human subject.
As Hellmann and Hellmann cites Leon Kass: "the physician must produce unswervingly the virtues of loyalty and fidelity to his patient."
%The physician has a duty towards the patient to do good and not cause harm.
It is simple to argue that this relationship should (and to a large extent also is) mandated primarily by deontological ethics.
The physician feel a \emph{moral} obligation to help her patients.
Close (some might argue closer) is the concept of \emph{virtue ethics}.
Beuchamp and Childress, cited in the paper "A virtue ethics approach to moral dilemmas in medicine", named five virtues applicable to the medical practitioner:
Trustworthiness, integrity, discernment, compassion and conscientiousness.
The physician that adheres to these five virtues should fulfill the common notion of a "good doctor".
Either if the main rationale stems from virtue or moral obligation, the physician should do what is best for her patient.
This extends naturally to all patients, and even though the physicians time and energy are limited resources, the "good doctor" would either disperse her time across all patients or refer the patients to another (good) doctor.

Now, the other of the dual roles of our physician comes into play, the scientist of the "physician-scientist".
I have used the word "patient" but I could just as well have used the term "research subject" to further separate the "care from the experiment".
It would be naive and a bit simple to say that the scientist disregard the well being of the patients.
On the contrary, the felt responsibility towards \emph{future} patients is potentially quite strong, with compassion and the intent to do no harm.
Although it is also natural to describe the scientific effort also in terms of deontological and virtue ethics, we are now moving to a realm where consequence-ethics is arguably more natural.
A smaller group (the research subjects) are sacrificing themselves for some "greater good", the potential of a better cure.
This is text-book utilitarianism.%, even though the treatment would be deemed in-effective, the medical knowledge has been expanded.
%the duty towards the expansion of knowledge and virtues governing good scientific conduct (honesty, curiosity perseverance etc. [cite Nature])
It is then tempting to accept RCTs based on consequencialism alone.
However, in my opinion it is far from straight forward, to illustrate:

Are the patients first and foremost patients or are they research-subjects?

Should we remove the medical doctors from the trial?

The patients themself - are they just means to justify our ends?

Written consent - yes, but do the patients know that?

An important aspect in an RCTs is something called "power", or "statistical power".
One conducts an analysis before the RCT start and calculates the needed participants in the study to reveal an effect.
The results from an "underpowered" RCT should be considered "scientifically worthless" (Halpern citing Altmann). 
To call them 
 
%There are numerous advantages that comes with randomization.
%First of all, it eliminates bias by trying to remove systematic differences between groups.
%A second, and a bit more complicated: 
%When we interpret results from clinical research, we use statistical theory which is based on random sampling to infer information of a larger group (in theory all patients) from relatively small group (the research subjects). (cite:altman).

%This is a more "broad" definition of invasive than commonly used.
%The last one is particularly vague and requires individual considerations.
%I will not for the sake of discussion limit the illness or ailment that is treated, for now it could be anything ranging from cosmetic to life-saving.






%\subsection*{Contemporary views}
%How is RCTs generally regarded?
%\subsection*{Contemporary discussions (laymen, the community)}
%What do the media and the general public think?
%\subsection*{Contemporary discussions (medical community)}
%How is regarded by the medical professionals, doctors and other scientists?
%\subsection*{Contemporary discussions (ethical scholars)}
%Are there a philosophical discussion? Have they reached any conclusions? Are they correct?
%\subsection*{In the frameworks of "big" ethical schools}
%Who have the biggest problem with RCTs, and should we care?
%
%
%\section*{Ethical resolutions?}
%\subsection*{Solutions to the problems (if any)}
%Is it all hopeless?
%\section*{Conclusion (possibly vague)}
%
%\section*{Own notes}
%
%\begin{enumerate}
%
%\item
%  The example with the kidney is interesting, as in "is there a difference to take out a kidney 
%  and throw it in the thrash, to investigate of whether or not a kidney "can" be extracted in contrast
%  it to take it out and put it into a needing receipient?
%\item
%  Separation between research and treatment - do we need the treatment aspect as it is not the same - failure to acknowledge that can lead to "very bad things"
%  But, also important, the complete separation can also lead to very bad things
%\item
%  Wacky though: suffering introduced (not death, or, why not death?) for some greater good - can it be drawn a line between "holocaust" and todays practice? 
%  If not, we have a problem.
%  
%  \end{enumerate}
%

\bibliography{test}

  %\begin{thebibliography}{9}
  %
  %  \bibitem{RN2}
  %Bolland, Mark J. and Avenell, Alison and Gamble, Greg D. and Grey, Andrew
  %Systematic review and statistical analysis of the integrity of 33 randomized controlled trials
  %Neurology
  %87
  %23
  %2391-2402
  %2016
  %
  %\bibitem{RN3}
  %Dunn, P.
  %James Lind (1716-94) of Edinburgh and the treatment of scurvy
  %Archives of Disease in Childhood Fetal and Neonatal Edition
  %76
  %1
  %F64-F65
  %1997
  %
  %\bibitem{RN1}
  %Savulescu, Julian and Wartolowska, Karolina and Carr, Andy
  %Randomised placebo-controlled trials of surgery: ethical analysis and guidelines
  %Journal of Medical Ethics
  %42
  %12
  %776-783
  %2016
  %
  %\end{thebibliography}
 

\end{document}

